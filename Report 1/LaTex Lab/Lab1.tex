\documentclass{article}
\usepackage{amsmath}   % for mathematical equations
\usepackage{graphicx}  % for including graphs and images
\usepackage{float}     % to force figure placement
\usepackage{caption}   % for figure captions
\usepackage{siunitx}   % for units formatting
\usepackage{hyperref}  % for links

\title{Lab Report 1: Lab 1 - 3}
\author{Enrique Rivera Jr \\ Marcus}
\date{\today}

\begin{document}
\maketitle

\begin{center}
\section*{Lab 1}
\end{center}


\subsection*{Abstract}
In this lab, we conducted a series of basic measurements using an oscilloscope and multimeter. The experiments focused on measuring open and closed circuit voltages, constructing voltage dividers, and using a function generator to produce and analyze waveforms. The results were compared to theoretical calculations, and graphical analyses were performed.

\section{Introduction}
This lab introduces basic electronic measurement techniques using a multimeter and oscilloscope. The objective is to measure DC and AC voltages, understand the concept of output impedance, and analyze waveforms using a function generator. By constructing simple circuits, we will learn how to measure voltage, current, and impedance, and observe the behavior of waveforms under different conditions.

\subsection{Methods}
\subsection{Breadboard Layout and Measurement}
Using a multimeter in buzzer mode, we first determined the internal connections of a breadboard. The multimeter's buzzing feature was used to identify connected holes. A simple diagram of the breadboard layout was made (Figure \ref{fig:breadboard}).

\begin{figure}[H]
    \centering
    \includegraphics[width=0.5\textwidth]{breadboard_diagram.png} % Replace with actual image path
    \caption{Diagram of breadboard connections.}
    \label{fig:breadboard}
\end{figure}

\subsubsection{Open and Closed Circuit Measurements}
The open circuit was constructed as shown in Figure \ref{fig:opencircuit}. The DC power supply was set to 5V, and the voltage was measured across the open circuit using the multimeter. Next, a closed circuit was constructed, and the current was measured using the ammeter function.

\begin{figure}[H]
    \centering
    \includegraphics[width=0.4\textwidth]{opencircuit_diagram.png} % Replace with actual image path
    \caption{Open circuit setup.}
    \label{fig:opencircuit}
\end{figure}

The output impedance $Z_{out}$ was calculated using the formula:
\[
Z_{out} = \frac{V_{\text{open}}}{I_{\text{closed}}}
\]

\subsubsection{Voltage Divider}
A voltage divider was constructed using two resistors of equal value (10k$\Omega$ each), and the output voltage was measured across the second resistor. The theoretical voltage was calculated using the voltage divider formula:
\[
V_{out} = V_{in} \times \frac{R_2}{R_1 + R_2}
\]

Both the measured and calculated voltages were recorded.

\subsubsection{Function Generator and Oscilloscope}
The output of a function generator was connected to an oscilloscope. A 1kHz sine wave with 2V peak-to-peak was generated. The waveform was observed and recorded on the oscilloscope screen. The effect of adding a 50$\Omega$ terminator was noted, and both the terminated and unterminated waveforms were sketched and compared.

\subsection{Results}
\subsubsection{Breadboard Layout}
The following diagram shows the internal connections of the breadboard. The multimeter was used to map out which holes were connected.

% Add breadboard diagram here

\subsubsection{Open and Closed Circuit Measurements}
The measured open circuit voltage was \SI{5}{V} and the closed circuit current was \SI{0.5}{A}. Using these values, the output impedance was calculated to be:
\[
Z_{out} = \frac{5 \text{V}}{0.5 \text{A}} = 10 \, \Omega
\]

\subsubsection{Voltage Divider Results}
The measured output voltage for the voltage divider was \SI{2.5}{V}, matching the theoretical calculation.

\begin{table}[H]
\centering
\caption{Voltage Divider Results}
\begin{tabular}{|c|c|c|}
\hline
$V_{in}$ (V) & $V_{out}$ (Measured) (V) & $V_{out}$ (Calculated) (V) \\ \hline
5            & 2.5                      & 2.5                        \\ \hline
\end{tabular}
\end{table}

\subsubsection{Function Generator and Oscilloscope}
The following waveforms were observed using the oscilloscope for the function generator output.

% Include Python-generated graphs here
\begin{figure}[H]
    \centering
    \includegraphics[width=0.6\textwidth]{waveform_terminated.png}  % Replace with actual path
    \caption{Waveform with terminator connected.}
    \label{fig:terminated_waveform}
\end{figure}

\begin{figure}[H]
    \centering
    \includegraphics[width=0.6\textwidth]{waveform_unterminated.png}  % Replace with actual path
    \caption{Waveform without terminator connected.}
    \label{fig:unterminated_waveform}
\end{figure}

\subsection{Conclusion}
In this lab, we successfully measured voltages and currents for open and closed circuits, calculated output impedance, and analyzed waveforms using an oscilloscope. The voltage divider results were consistent with theoretical predictions, and the function generator output was analyzed with and without a terminator. This lab provided a foundational understanding of electronic measurement techniques.


\begin{center}
    \section*{Lab 2}
\end{center}


\begin{center}
    \section*{Lab 3}
    \end{center}

\end{document}
