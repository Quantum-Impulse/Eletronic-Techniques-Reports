\documentclass{article}
\usepackage{amsmath}   % for mathematical equations
\usepackage{graphicx}  % for including graphs and images
\usepackage{float}     % to force figure placement
\usepackage{caption}   % for figure captions
\usepackage{siunitx}   % for units formatting
\usepackage{hyperref}  % for links

\title{Lab Report 2: Lab 4-6}
\author{Enrique Rivera Jr \\ Marcus}
\date{\today}

\begin{document}
\maketitle

\section*{Lab 4: Diodes}

\subsection*{Important Concepts}
- \textbf{Diode}: A semiconductor device that allows current to flow in one direction only. It has two terminals: an anode (+) and a cathode (-).
\newline
- \textbf{Voltage Divider}: A circuit used to create a voltage less than or equal to the input voltage.
\newline
- \textbf{Diode Forward Voltage Drop}: Typically around 0.7V for silicon diodes.
\newline
- \textbf{Zener Diode}: Designed to operate in reverse bias, with a specified breakdown voltage.

\subsection*{Equations}
- \textbf{Diode Equation}:
\[
I = I_s \left( e^{\frac{V}{nV_T}} - 1 \right)
\]
- \textbf{Voltage Divider}:
\[
V_{out} = V_{in} \cdot \frac{R_2}{R_1 + R_2}
\]
- \textbf{Ripple Voltage}:
\[
V_{ripple} = \frac{I_{load}}{fC}
\]

\subsection{FYR Questions}
1. \textbf{Diode Identification and Voltage Drop:}

\begin{figure}[H]
    \centering
    \includegraphics[width=0.9\textwidth]{./img/Lab4_1.png}
    \caption{Drawing of Diode}
    \label{fig:graph1}
\end{figure}


\newline
\newline

2. \textbf{Diode Circuits: }The diode behaves like a switch; it is on when forward-biased (Vin > 0.7V), and off otherwise.

The Yellow line represents the Vin, the Blue line represents the Vout.

\begin{figure}[H]
    \centering
    \includegraphics[width=0.9\textwidth]{./img/other/Lab4_Circuits.png}
    \caption{Circuits For This Question}
    \label{fig:graph2}
\end{figure}

\begin{figure}[H]
    \centering
    \includegraphics[width=0.9\textwidth]{./img/Lab4_2_1.png}
    \caption{Circuit 1 Results}
    \label{fig:graph3}
\end{figure}

\textbf{Description:} A forward-biased diode in series with a resistor.

\textbf{Expected Behavior:} The diode will conduct (be "on") when the input voltage 
Vin is greater than the forward voltage of the diode 
(about 0.7V for a silicon diode). When the input drops below 0.7V, the diode 
will be off, and Vout will be near zero.

\textbf{Oscilloscope Data:} In the graph, you can see that the yellow waveform 
(Vin) is sinusoidal, and the blue waveform (Vout) shows clipping at 
approximately 0.7V, confirming that the diode only conducts when Vin>0.7V.

\begin{figure}[H]
    \centering
    \includegraphics[width=0.9\textwidth]{./img/Lab4_2_2.png}
    \caption{Circuit 2 Results}
    \label{fig:graph4}
\end{figure}

\textbf{Description:} A reverse-biased diode in series with a resistor.

\textbf{Expected Behavior:} In this case, the diode will not conduct under normal forward voltage conditions, as it is reverse-biased. Hence, 
Vout will be almost zero throughout the entire cycle of = Vin.

\textbf{Oscilloscope Data:} The second graph shows that the blue waveform (Vout) is almost flat, indicating the diode is not conducting, as expected for a reverse-biased diode.

\begin{figure}[H]
    \centering
    \includegraphics[width=0.9\textwidth]{./img/Lab4_2_3.png}
    \caption{Circuit 3 Results}
    \label{fig:graph5}
\end{figure}

\textbf{Description:} A forward-biased diode, with the diode grounded.

\textbf{Expected Behavior:} Similar to Circuit 1, but this time, the diode is connected to the ground. The output 
Vout will be clamped at 0.7V during the positive cycle of 
Vin, and the diode will conduct when > 0.7V. Vin>0.7V. 

\textbf{Oscilloscope Data:} In the third graph, the blue waveform again shows clipping at around 0.7V, indicating that the diode is conducting when 
Vin>0.7V, and is off otherwise.

\begin{figure}[H]
    \centering
    \includegraphics[width=0.9\textwidth]{./img/Lab4_2_4.png}
    \caption{Circuit 4 Results}
    \label{fig:graph6}
\end{figure}

\textbf{Description:} A reverse-biased diode, similar to Circuit 2 but with reversed polarity.

\textbf{Expected Behavior:} The diode will not conduct during the positive cycles of 
Vin, but it may conduct slightly during negative cycles if the reverse breakdown voltage is reached. However, for most practical situations, the output will be near zero.

\textbf{Oscilloscope Data:} The fourth graph shows minimal activity on the blue waveform (Vout), confirming that the diode remains off, as expected for a reverse-biased diode.s
\newline
\newline

3.\textbf{Voltage Clamp Circuit:} Adjusting the +1V shifted the output. The diode was **on** when Vin exceeded the forward voltage and bias.

\begin{figure}[H]
    \centering
    \includegraphics[width=0.9\textwidth]{./img/other/Lab4_3.png}
    \caption{Circuit For Question 3}
    \label{fig:graph7}
\end{figure}

\begin{figure}[H]
    \centering
    \includegraphics[width=0.9\textwidth]{./img/Lab4_3_1v.png}
    \caption{Circuit with +1V Shift. We can see that on Vout we are seeing a 1V shift will keeping the negative cycle.} 
    \label{fig:graph8}
\end{figure}

\begin{figure}[H]
    \centering
    \includegraphics[width=0.9\textwidth]{./img/Lab4_3_3.5v.png}
    \caption{Circuit with 3.5V Shift. We can see that on Vout we are seeing a 3.5V shift will keeping the negative cycle.}
    \label{fig:graph9}
\end{figure}
\newline
\newline

4.\textbf{Two Diode Circuit} Sketched regions where D1 and D2 were on and off, showing rectification.
\begin{figure}[H]
    \centering
    \includegraphics[width=0.9\textwidth]{./img/other/Lab4_4.png}
    \caption{Circuit for Question 4}
    \label{fig:graph10}
\end{figure}

\begin{figure}[H]
    \centering
    \includegraphics[width=0.9\textwidth]{./img/Lab4_4.png}
    \caption{Circuit Results}
    \label{fig:graph11}
\end{figure}

Explanation of Diode Behavior:
D1 (Forward Biased): When the input signal is positive, D1 conducts and the output voltage is clamped near 0.7V.
D2 (Forward Biased): When the input signal is negative, D2 conducts and the output is clamped near -0.7V.

Regions of On and Off States:
D1 is on during the positive half-cycle of the input waveform when Vin $\ge$ 0.7V.
D2 is on during the negative half-cycle of the input waveform when Vin$\le$-0.7V.
During these times, the output is clamped to the diode's forward voltage, limiting the maximum and minimum values of 
Vout.

5.\textbf{Ripple and DC} The plot of \( VDC/Vripple \) versus R indicated a smoother rectification with higher resistance.

\begin{figure}[H]
    \centering
    \includegraphics[width=0.9\textwidth]{./img/other/Lab4_5.png}
    \caption{Circuit For Question 5}
    \label{fig:graph12}
\end{figure}

\section*{Question 5}

We used a capacitor with a capacitance of \(C = 0.94 \, \mu F\) and resistances of \( R_1 = 106.4 \, k\Omega \) and \( R_2 = 48.7 \, k\Omega \). The circuit was set at a frequency of 1.5 kHz.

The ripple voltage \( V_{ripple} \) is defined as:

\[
V_{ripple} = V_{\text{max}} - V_{\text{min}}
\]

The DC voltage \( V_{DC} \) is given by:

\[
V_{DC} = \frac{V_{\text{max}} + V_{\text{min}}}{2}
\]

We recorded the following values for different resistances:

\begin{table}[H]
\centering
\begin{tabular}{|c|c|c|}
\hline
Filtering Resistance (R) & \( V_{ripple} \) (V) & \( V_{DC} \) (V) \\
\hline
100 k\(\Omega\) & 0.6 & 4.44 \\
50 k\(\Omega\) & 0.4 & 4.76 \\
10 k\(\Omega\) & 0.2 & 4.86 \\
5 k\(\Omega\)  & 0.8 & 4.2  \\
200 \(\Omega\) & 3.4 & 1.51 \\
500 \(\Omega\) & 3.0 & 2.32 \\
\hline
\end{tabular}
\caption{Measured ripple voltage \( V_{ripple} \) and DC voltage \( V_{DC} \) for different filtering resistances.}
\end{table}

\begin{figure}[H]
    \centering
    \includegraphics[width=0.9\textwidth]{./img/Lab4_5.png}
    \caption{Graph of \( VDC/Vripple \) versus R}
    \label{fig:graph13}
\end{figure}


6.\textbf{I-V Curve of Zener Diode}: The Zener diode conducted in reverse once the 5V breakdown voltage was reached.

\begin{figure}[H]
    \centering
    \includegraphics[width=0.9\textwidth]{./img/other/Lab4_6.png}
    \caption{Graph of \( VDC/Vripple \) versus R}
    \label{fig:graph14}
\end{figure}

We used a Zener diode with a forward voltage of 0.702V. The resistor was set at \( R = 47.6 \, k\Omega \). The goal was to limit the current to 50 mA.

From Ohm's law:

\[
R = \frac{V}{I} = \frac{5 \, \text{V}}{0.05 \, \text{A}} = 100 \, \Omega
\]

### Measured I-V Data:
The following table presents the measured current for different voltage values across the Zener diode.

\[
\begin{array}{|c|c|}
\hline
V \, (\text{Volts}) & I \, (\text{Amperes}) \\
\hline
1.0 & 2.6 \, \text{mA} \\
2.0 & 9.8 \, \text{mA} \\
3.0 & 19.6 \, \text{mA} \\
4.0 & 28.6 \, \text{mA} \\
5.6 & 36.1 \, \text{mA} \\
6.0 & 46.5 \, \text{mA} \\
7.0 & 55.6 \, \text{mA} \\
8.0 & 63.5 \, \text{mA} \\
9.0 & 73.1 \, \text{mA} \\
10.0 & 82.1 \, \text{mA} \\
15.0 & 125.1 \, \text{mA} \\
19.0 & 145.1 \, \text{mA} \\
\hline
\end{array}
\]

### Maximum Current:
The maximum current measured was 145.1 mA at 19V.

\begin{figure}[H]
    \centering
    \includegraphics[width=0.9\textwidth]{./img/Lab4_6.png}
    \caption{Graph of \( VDC/Vripple \) versus R}
    \label{fig:graph15} 
\end{figure}


\end{document}
